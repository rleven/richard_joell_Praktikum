\section{Durchführung}
\label{sec:Durchführung}

Zuerst werden die Uhr und der Schaltkreis, nach der Vorlage aufgebaut, sodass der Schaltkreis die Stoppuhr aktiviert sobald ein Lichtimpuls den Empfänger streift und erst stoppt, sobald der Empfänger das dritte mal gestreift wurde.
Daraus ergibt sich die Dauer einer Periode.
\\
Zuerst wird die Messung mit B-Feld vorgenommen.
Es werden für 10 verschiedene Stromstärken je 10 Messungen vorgenommen und in einer Tabelle festgehalten(\autoref{tab:taba}).
Mithilfe \autoref{eq:formel13} wird daraus das magnetische Moment m berechnet.
Anschließend wird der Strom für die Spulen ausgeschaltet und es wird eine Messung mit je 10 Messwerten vorgenommen, um das Torsionsmodul G zu bestimmen.
Zuletzt wird aus den Werten von G und E die Poisson'sche Querkontraktionszahl $\mu$ und das Kompressionsmodul Q berechnet.