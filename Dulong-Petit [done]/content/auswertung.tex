\newpage
\section{Auswertung}
\label{sec:Auswertung}

\subsection{Die Wärmekapazität des Kalorimeters}

Zunächst wird die Wärmekapazität des Kalorimeters bestimmt. Dies wird mit der Hilfe von folgender Gleichung gemacht:

\begin{equation}
    Q_1 = c_km_k(T_k-T_m)
\end{equation}
und 
\begin{equation}
    Q_2 = (c_wm_w+c_gm_g)(T_m-T_w)
\end{equation}
woraus sich für die Wärmekapazität des Kalorimeters ergibt:
\begin{equation}
    c_gm_g = \frac{c_wm_k(T_k-T_m)-c_wm_w(T_m-T_w)}{T_m-T_w}
    \label{eq:cgmg}
\end{equation}
Und für \(c_k\):
\begin{equation}
    c_k = \frac{(c_wm_w+c_gm_g)(T_m-T_w)}{m_k(T_k-T_m)}
    \label{eq:ck}
\end{equation}
Wobei \(T_k\) die Temperatur des Wassers im erhitzten Gefäß, \(T_w\) die Temperatur des Wassers im Kalorimeters und \(T_m\) die Mischungstemperatur ist. \(c_w\) ist die Wärmekapazität des Wassers, welche 4,18 J/gK beträgt und \(c_gm_g\) die Wärmekapazität des Kalorimeters.
Hier die Werte:
%y=k,x=w
\begin{itemize}
  \centering
  \item[] \(T_w\) = 21,3°C = 294,5 K\\
  \item[] \(T_k\) = 80,0°C = 353,15 K\\
  \item[] \(T_m\) = 47,2°C = 320,35 K\\
  \item[] \(m_w\) = 300,03 g\\
  \item[] \(m_k\) = 290,00 g\\ 
\end{itemize}
Somit ergibt sich für die Wärmekapazität des Kalorimeters:

\begin{equation}
    c_gm_g = 283.99 \frac{J}{K}
\end{equation}
\newpage
\subsection{Die Wärmekapazitäten verschiedener Stoffe}
Zuerst wird die Wärmekapazität von Zinn anhand 3 Messungen bestimmt, mittels \autoref{eq:ck} und \autoref{eq:cgmg}.
  \begin{table}[htp]
    %Messwerte aus a)
    \centering
    \caption{Die Werte von dem Experiment mit Zinn.}
    \label{tab:tabZn}
    \begin{tabular}{c c c c}
      \toprule
      Daten & Messung 1 & Messung 2 & Messung 3\\
      \midrule
            \(m_w\)[g] & 616.50 & 616.59 & 617.60\\
            \(m_k\)[g] & 203.04 & 203.04 & 203.04\\
            \(T_w\)[K] & 294.50 & 294.85 & 296.85\\
            \(T_k\)[K] & 353.15 & 353.15 & 353.15\\
            \(T_m\)[K] & 295.95 & 296.15 & 297.95\\
      \bottomrule
            \(c_{k(Zinn)}\)[J/gK] & 0.36 & 0.32 & 0.28\\
    \end{tabular}
  \end{table}

Der gemittelte Wert für $c_{k(Zinn)}$ liegt dabei bei $\bar{c_k} = 0.32\pm 0.04\ \frac{J}{gK}$.
Alle Mittelungen und deren Standardabweichungen wurden mit \autoref{eq:std} bestimmt.\\
\newline
Nun wird $c_k$ für Aluminium bestimmt(\autoref{tab:tabAl}).
  \begin{table}[htp]
    %Messwerte aus a)
    \centering
    \caption{Die Werte von dem Experiment mit Aluminium.}
    \label{tab:tabAl}
    \begin{tabular}{c c c c}
      \toprule
      Daten & Messung 1 & Messung 2 & Messung 3\\
      \midrule
            \(m_w\)[g] & 593.23 & 616.68 & 637.91\\
            \(m_k\)[g] & 158.83 & 158.83 & 158.83\\
            \(T_w\)[K] & 294.75 & 294.45 & 294.35\\
            \(T_k\)[K] & 353.15 & 353.15 & 353.15\\
            \(T_m\)[K] & 298.15 & 297.65 & 297.15\\
      \bottomrule
            \(c_{k(Aluminium)}\)[J/gK] & 1.08 & 1.03 & 0.93\\
    \end{tabular}
  \end{table}

Der gemittelte Wert für $c_{k(Aluminium)}$ liegt bei $\bar{c_k} = 1.01\pm 0.08\ \frac{J}{gK}$.\\
\newline
Es wurde nur eine Messung zur Wärmekapazität von Graphit aufgenommen(\autoref{tab:tabC}).
  \begin{table}[htp]
    %Messwerte aus a)
    \centering
    \caption{Die Werte von dem Experiment mit Graphit}
    \label{tab:tabC}
    \begin{tabular}{c c}
      \toprule
      Daten & Werte\\
      \midrule
            \(m_w\)[g] & 623.96\\
            \(m_k\)[g] & 105.00\\
            \(T_w\)[K] & 294.55\\
            \(T_k\)[K] & 353.15\\
            \(T_m\)[K] & 296.95\\
      \bottomrule
            \(c_{k(Graphit)}\)[J/gK] & 1.18\\
    \end{tabular}
  \end{table}
\newpage
Die Abweichungen von den Literaturwerten betragen für Aluminium 12,72\%.\\
Für Zinn 39,13\%.\\
Für Graphit dann noch 49,37\%.

\subsection{Die Molwärmen der Stoffe}

Die Molwärme der Stoffe wird mithilfe von \autoref{eq:cp} bei einer Temperatur von $80°C = 353.15K$ berechnet und in \autoref{tab:tabMol} aufgeführt.

\begin{table}[htbp]
    \centering
    \caption{Zu sehen sind die berechneten Molwärme der Stoffe und die Abweichungen.\cite{warm}}
    \label{tab:tabMol}
    \begin{tabular}{c c c c}
        \toprule
        Elemente & Molwärme [$\frac{J}{mol\cdot K}$]& spezifische Wärmekapazität [$\frac{J}{gK}$] & Abweichungen der $\bar{c_k}$ [\%]\\
        \midrule
        Zinn & $35.906\pm 4.748$ & $0.32\pm 0.04$ & $28.13\pm 8.98$\\
        Aluminium & $25.954\pm 2.16$ & $1.01\pm 0.08$ & $11.39\pm 7.02$\\
        Graphit & $14.124$ & $1.18$ & $39.41$\\
        \bottomrule
    \end{tabular}
\end{table}