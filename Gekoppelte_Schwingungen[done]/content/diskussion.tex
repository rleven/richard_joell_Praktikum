\section{Diskussion}
\label{sec:Diskussion}

Das Verhältnis von Schwebungsfrequenz und Schwingungsfrequenz weicht gemittelt um 14.55\% von dem gemittelten errechneten Wert ab.
Der plausibelste Grund dafür ist die ungenaue Bestimmung der Maxima am Oszillographen, sowie die Innenwiderstände der Geräte und Kabel.
\\
Ähnlich verhält er sich auch mit den Lissajou-Figuren. Die Messungen am Oszillographen sind nur begrenzt genau.
Daher weichen auch hier die Messwerte von den Berechneten leicht ab. 
Trotzdem ist eine Genauigkeit von ca. 5\% ausreichend, um den gekoppelten Schwingkreis damit zu beschreiben.
Interne Widerstände der Verbindungskabel, sowie die Magnetfelder der Spulen beeinflussen zum Teil den Stromfluss und sorgen für Fluktuationen.
\\
Bei der Messung des Stroms im Schwingkreis sind nur die Messungen am Oszillographen fehlerbehaftet, doch dies hat eine sehr geringe Auswirkung auf das Resultat.
Da sowieso die berechneten Werte für \(|L|\) von 0.00999 auf 0.01 gerundet wurden und die Ausgangsspannung \(|U|\) konstant bei 8V lag, ist eine Stromstärke von 0.08A recht genau.

