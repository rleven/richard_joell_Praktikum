\section{Diskussion}
\label{sec:Diskussion}



Die theoretischen Schwingungsverhältnisse weisen gegenüber den gemessenen große Differenzen auf.
Im Durchschnitt liegen die gemessenen Werte weit unter den theoretischen.
Das ist auf systematische Fehler zurückzuführen, ebenso wie die ungenaue Bestimmung der Maxima am Oszillographen, sowie die Innenwiderstände der Geräte und Kabel.
\\
Ähnlich verhält er sich auch mit den Lissajous-Figuren. Die Messungen am Oszillographen sind nur begrenzt genau.
Daher weichen auch hier die Messwerte von den Berechneten leicht ab, wie man an \autoref{fig:plot2} erkennen kann.
Die theoretischen Werte beider gemessenen Größen kommen an diese sehr genau ran.
\\
Bei der Messung des Stroms im Schwingkreis sind nur die Messungen am Oszillographen fehlerbehaftet, doch dies hat eine sehr geringe Auswirkung auf das Resultat.
Dabei liegt der Wert von \(L\) in derselben Größenordnung, gerundet beim Wert 0.01.

%"Wer legt das fest?" Gering.
%Tabelle mit Abweichungen machen und dann darauf referieren
%Tabellen 3 und 4 + Abweichungen davon fusionieren.
%Der Wert von L liegt gerundet beim Wert 0.01 + Einheit - recht genau.
%Auf alle Abbildungen eingehen: passen Theoriekurve und Messwerte? Selbe Größenordnung, Verhalten von Abweichung + Gründe dafür? Wenn Messkurve immer größer/kleiner Theoriekurve, systematische Fehler. Ansonsten auch andere Fehler. 
