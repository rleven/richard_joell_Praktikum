\documentclass[captions=tableheading]{scrartcl}
\usepackage[aux]{rerunfilecheck}

\usepackage{fontspec}
\usepackage[main=ngerman]{babel}
\usepackage[unicode]{hyperref}
\usepackage{bookmark}
\usepackage{booktabs}
\usepackage{amsmath}
\usepackage{amssymb}
\usepackage{mathtools}
\usepackage{pdfpages}
\setmainfont{Libertinus Serif}
\subject{Versuchsnummer: 803}
\title{Das Hook'sche Gesetz}
\author{Richard Leven \and Joell - D. Jones} 
\date{
    Durchführung: 05.11.2019\\
    Abgabe: 12.11.2019
}
\publishers{TUDortmund -Fachschaft Physik}
\begin{document}
\maketitle
\section{Der Versuch}
\subsection{Versuchsbeschreibung}
Es ist ein Lineal horizontal angelegt. Darauf befindet sich ein Pfeil, welcher mit einem Faden verbunden ist.\\ Dieser Faden geht von einer Feder aus und wird von einer Rolle umgelenkt. Auf dem Bildschirm wird die ausgeübte Kraft angezeigt.

\subsection{Versuchsdurchführung}

Der Pfeil wird nun auf dem Lineal nach rechts gezogen. Der Faden zieht dementsprechend an der Feder und übt eine Gegenkraft aus, welche dann vom Laptop angezeigt wird.\\ Dies wird dann entsprechend in einer Tabelle dokumentiert.

\newpage
\section{Auswertung}
\subsection{Wertetabelle}
    
    \begin{table}[htp]
        \centering
        \caption{Tabelle der Messwerte}
        \label{tab:some_data}
        \begin{tabular}{c c c}
            \toprule
            $\Delta x [m]$ & $F [N]$ & $D$\\
            \midrule
            0.05 & 0.15 & 3.00 \\
            0.10 & 0.29 & 2.90 \\
            0.15 & 0.44 & 2.93 \\
            0.20 & 0.59 & 2.95 \\
            0.25 & 0.74 & 2.96 \\
            0.30 & 0.89 & 2.96 \\
            0.35 & 1.04 & 2.97 \\
            0.40 & 1.19 & 2.98 \\
            0.45 & 1.34 & 2.98 \\
            0.50 & 1.49 & 2.98 \\
            \bottomrule
        \end{tabular}
    \end{table}

\section{Berechnung}

\subsection{Der Mittelwert}

Sei N die Anzahl der aufgezeichneten Messwerte, in dem Fall also 10.

Mit den gegebenen Werten lässt sich der Mittelwert wie folgt berechnen:

\begin{equation}
\overline{D} = \frac{1}{N}\sum_{k=1}^N D_{k}
\end{equation}
mit 
\begin{equation}
D_{k} = \frac{\Delta x_{k}}{F_{k}}
\end{equation}
.
\newpage
\subsection{Auswertung}

Rechnet man mit den Werten der Tabelle und nutzt die oben gegebene Formel, dann ergibt sich für den Mittelwert von D =
2.96 kg/s².

\subsection{Ausgleichsgerade}

Zur Berechnung der Ausgleichsgerade haben wir in unserem Python-Programm den Befehl linregress benutzt.
Dieser gibt uns die Steigung sowie den Schnittpunkt mit der y-Achse an.\\
Alle Punkte aufgetragen und die Ausgleichsgerade hindurchgelegt, sieht der Plot wie folgt aus: \\
\\ \\ \\ \\ \\ \\ \\ \\ \\ \\ \\ \\ \\ \\ \\ \\ \\ \\ \\
\begin{figure}
\includepdf[scale=0.45, offset= 0mm 20mm]{build/Ausgleichsgerade}
\end{figure}


Mit den gemessenen Werten gibt der Befehl linregress für die Steigung der Geraden den Wert 2.99 aus.
Dies ist der Wert der Federkonstante in kg/s².\\
Der mit dem Mittelwert errechnete Wert weicht somit um 0.03 kg/s² davon ab.
%.
\end{document}