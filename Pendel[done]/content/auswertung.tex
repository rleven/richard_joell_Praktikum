\section{Auswertung}
\label{sec:Auswertung}

Die gemessenen Schwingungsdauern der einzelnen Pendel sind in \autoref{tab:einpendel} zu sehen, ebenso wie dessen korrekte Dauer für eine enizelne Schwingung.

\begin{table}
  \centering
  \caption{Schwingungsdauern der einzelnen Pendel ohne Feder in $[s]$.}
  \resizebox{\textwidth}{!}{%
  \csvreader[tabular=c|c|c|c,
  head=false, 
  table head= Messung für $T_1$ & Messung für $T_2$ & Schwingungdauer von $T_1$ & Schwingungsdauer von $T_2$ \\\midrule,
  late after line= \\]
  {tabelle1.csv}{1=\eins, 2=\zwei, 3=\drei, 4=\vier}{$\num{\eins}$ & $\num{\zwei}$ & $\num{\drei}$ & $\num{\vier}$}}
  \label{tab:einpendel}
\end{table}

Die Werte sind sehr nah beieinander, sodass die Pendellänge bei beiden gleich bleiben konnte.
Aus den gesammelten Werten ergibt sich ein gemittelter Wert für $T_1$ und $T_2$:
\begin{equation}
  T_1 = 1.547\ s \pm 0.055\ s \quad\textrm{und}\quad T_2 = 1.498\ s \pm 0.049\ s
\end{equation}
\subsection{Gleichphasige Schwingung}
Die gleichphasige Schwingung wurden bei $l_1 = 58\ cm\pm 0.1\ cm$ und $l_2 = 83\ cm\pm 0.1\ cm$ gemessen.

\begin{table}
  \centering
  \caption{Schwingungsdauern der gekoppelten, gleichphasigen Pendel in $[s]$.}
  \resizebox{\textwidth}{!}{%
  \csvreader[tabular=c|c|c|c,
  head=false, 
  table head= Messung für $l_1$ & Messung für $l_2$ & Schwingungsdauer bei $T_{l_1}$ & Schwingungsdauer bei $T_{l_2}$ \\\midrule,
  late after line= \\]
  {tabelle2.csv}{1=\eins, 2=\zwei, 3=\drei, 4=\vier}{$\num{\eins}$ & $\num{\zwei}$ & $\num{\drei}$ & $\num{\vier}$}}
  \label{tab:gleichpendel}
\end{table}

Die \autoref{tab:gleichpendel} zeigt die Dauer einer Schwingung bei verschiedenen Pendellängen.\\
Die gemittelte Schwingungsdauer beträgt für $l_1 = 58\ cm\pm 0.1\ cm$ $T_{l_1} = 1.540\ s\pm 0.007\ s$ und für $l_2 = 83\ cm\pm 0.1\ cm$ $T_{l_2} = 1.788\ s\pm 0.007\ s$.
\newpage
Gemäß \autoref{eq:lol1} ergibt der theoretische Wert bei\\ $l_{1}$: $1.528\ s\pm0.001\ s$\\ $l_{2}$: $1.828\ s\pm0.001\ s$\\
Die Fehlerrechnung wurde gemäß der Gauß'schen Fehlerfortpflanzung aus \autoref{sec:gauß} berechnet.\\
Die Abweichung vom gemittelten, gemessenen $T_{l_1}$ zum theoretischen Äquivalent beträgt ca. 0.779\%.
Die Abweichung bei $T_{l_2}$ liegt bei ca. 2.188\%.

\subsection{Gegenphasige Schwingung}
Bei der gegenphasigen Schwingung wurden ebenfalls die Längen $l_1, \ l_2$ benutzt.

\begin{table}
  \centering
  \caption{Schwingungsdauern der gekoppelten, gegenphasigen Pendel in $[s]$.}
  \resizebox{\textwidth}{!}{%
  \csvreader[tabular=c|c|c|c,
  head=false, 
  table head= Messung für $l_1$ & Messung für $l_2$ & Schwingungsdauer bei $l_1$ & Schwingungsdauer bei $l_2$ \\\midrule,
  late after line= \\]
  {tabelle3.csv}{1=\eins, 2=\zwei, 3=\drei, 4=\vier}{$\num{\eins}$ & $\num{\zwei}$ & $\num{\drei}$ & $\num{\vier}$}}
  \label{tab:gegenpendel}
\end{table}

Die gemittelten Schwingungsdauern aus \autoref{tab:gegenpendel} sind $T_{l_1} = 1.356\ s\pm 0.008\ s$ und $T_{l_2} = 1.670\ s\pm 0.005\ s$.

\subsection{Gekoppelte Schwingung}
Die Messwerte zur Schwingungsdauer $T$ und zur Schwebungsdauer $T_S$ für beide Pendellängen sind in \autoref{tab:koppendel} aufgelistet.
\newpage
\begin{table}
  \centering
  \caption{Schwingungsdauern der gekoppelten Pendel in $[s]$.}
  \csvreader[tabular=c|c|c|c,
  head=false, 
  table head= $T$ für $l_1$ & $T$ für $l_2$ & $T_S$ bei $l_1$ & $T_S$ bei $l_2$ \\\midrule,
  late after line= \\]
  {tabelle4.csv}{1=\eins, 2=\zwei, 3=\drei, 4=\vier}{$\num{\eins}$ & $\num{\zwei}$ & $\num{\drei}$ & $\num{\vier}$}
  \label{tab:koppendel}
\end{table}

Die gemittelte Schwingungsdauer $T$ beträgt für\\ $l_1$: $T = 1.477\ s\pm 0.015\ s$\\ $l_2$: $T = 1.842\ s\pm 0.029\ s$\\
Die gemittelte Schwebungsdauer $T_S$ beträgt für\\ $l_1$: $T_S = 12.723\ s\pm 0.116\ s$\\ $l_2$: $T_S = 22.661\ s\pm 0.349\ s$

\subsection{Berechnung der Kopplungskonstante $K$}
\label{sec:ksec}
Die jeweiligen Frequenzen zu den Schwingungdauern lassen sich mit $\frac{2\pi}{T}$ berechnen.\\
Die Schwebungsfrequenz berechnet sich mit \autoref{eq:lol3}.
Demnach lauten die Frequenzen zu den gemittelten Schwingungsdauern:

\begin{table}
  \centering
  \caption{Frequenzen für die gemessenen Schwingungsdauern.}
  \begin{tabular}{c c c}
    \toprule
    $\omega$ & Werte für $l_1$ in $Hz$ & Werte für $l_2$ in $Hz$\\
    \midrule
    $\omega_+$ & $4.08\pm 0.019$ & $3.514\pm 0.014$\\
    $\omega_-$ & $4.634\pm 0.027$ & $3.762\pm 0.011$\\
    $\omega_S$ & $0.494\pm 0.005$ & $0.277\pm 0.004$\\
  \end{tabular}
  \label{lab:omgtab}
\end{table}

Nach der \autoref{eq:Kform} liegt der Wert der Kopplungskonstante bei:
\begin{equation}
  K_{l_1} = \frac{(4.634\pm 0.027)^2\ -\ (4.08\pm 0.019)^2}{(4.634\pm 0.027)^2\ +\ (4.08\pm 0.019)^2} = 0.067\pm 0.008
\end{equation}
und
\begin{equation}
  K_{l_2} = \frac{(3.762\pm 0.011)^2\ -\ (3.514\pm 0.014)^2}{(3.762\pm 0.011)^2\ +\ (3.514\pm 0.014)^2} = 0.068\pm 0.005
\end{equation}
Die Werte für dieselbe Feder haben eine Abweichung von ca. $1.47\%$ voneinander.\\
Gemittelt hat $K$ den Wert $0.0675\pm 0.0047$.

\subsection{Vergleich der theoretischen $T_S$ und gemessenen.}
Gemäß \autoref{eq:Tpose} ergibt die theoretische Schwebungsfrequenz:
\begin{equation}
  T_{S1} = \frac{1.54\cdot 1.356}{1.54\ -\ 1.356}\ s = 11.349\ s \quad\textrm{für}\quad l_1
\end{equation}
und
\begin{equation}
  T_{S2} = \frac{1.788\cdot 1.670}{1.788\ -\ 1.670}\ s = 25.305\ s \quad\textrm{für}\quad l_2
\end{equation}
Die Abweichung zu $T_{S1}$ zum gemessenen Wert beträgt ca. 10.8\%.\\
Die Abweichung zu $T_{S2}$ zum gemessenen Wert beträgt ca. 10.4\%.\\