\section{Diskussion}
\label{sec:Diskussion}

\subsection{Gleichphasige Schwingung}
Die gemessenen und errechneten Werte für die Schwingungsdauern bei beiden Pendellängen lagen sehr nah beieinander.\\
Die Feder hat somit keine Kraft bei der Messung übertragen, wie es beabsichtigt war.
\subsection{Gegenphasige Schwingung}
Die thoretischen Werte für $T_-$ brauchen zum Berechnen die Kopplungskonstante $K$, die in \autoref{sec:ksec} gemittelt wurde und ca bei 0.01 liegt.
Der daraus berechnete, theoretische Wert für $T_-$ weicht nur ca. 10.38\% bei $l_1$ und ca. 7.73\% bei $l_2$ vom gemessenen, gemittelten ab.
Ein Grund dafür ist, dass sich die Pendel per Hand einfacher in eine gleiche Phase bringen lassen, als in eine entgegengesetzte.
\subsection{Gekoppelte Schwingung}
Die gemessenen Werte der Schwebungsdauern entsprechen nicht den theoretischen Werten, da diese bei beiden Längen um etwa das 13-fache größer sind.
Das bezieht sich nicht auf $T_{S1}$ und $T_{S2}$, diese Werte wurden aus Messwerten errechnet.
Das lässt sich nur durch Messunsicherheiten und einer Ungenauigkeit oder Fehlerbehaftung im Aufbau des Experiments erklären.\\
Auch unter Verwendung der nicht gemittelten Werte für $K$ ergeben sich Abweichung in ähnlicher Größenordnung.
\subsection{Schwebungsdauer}
Eine ca 10\%-ige Abweichung von den gemessenen Werten bei beiden Längen ist hier plausibel, da die Schwebung per Augenmaß festgestellt wurde und es weitere statistische Messfehler gab.