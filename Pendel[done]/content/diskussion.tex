\section{Diskussion}
\label{sec:Diskussion}

\subsection{Gleichphasige Schwingung}
Die gemessenen und errechneten Werte für die Schwingungsdauern bei beiden Pendellängen liegen sehr nah beieinander.\\
Die Feder hat somit keine Kraft bei der Messung übertragen, wie es beabsichtigt war.
\subsection{Gegenphasige Schwingung}
Die thoretischen Werte für $T_-$ brauchen zum Berechnen die Kopplungskonstante $K$, die in \autoref{sec:ksec} gemittelt wurde und bei $0.0675\pm 0.0047$ liegt.
Der daraus berechnete, theoretische Wert für $T_-$, nach \autoref{eq:lol2}, weicht nur ca. $10.67\%\pm 0.53\%$ bei $l_1$ und ca. $7.99\%\pm 0.28\%$ bei $l_2$ vom gemessenen, gemittelten ab.
Ein Grund dafür ist, dass sich die Pendel per Hand einfacher in eine gleiche Phase bringen lassen, als in eine entgegengesetzte.
\subsection{Gekoppelte Schwingung}
Die gemessenen Werte der Schwebungsdauern entsprechen nicht den theoretischen Werten, da diese bei beiden Längen um etwa das 21-fache größer sind.
So ergibt sich für die Länge $l_1$ eine theoretische Schwebungsdauer von ca $231.95\ s\pm 32.8\ s$, aber die aus den Messwerten errechnete Schwebungsdauer liegt bei $11.349\ s$. 
Das lässt sich nur durch Messunsicherheiten, Reibung und einer Ungenauigkeit oder Fehlerbehaftung im Aufbau des Experiments erklären.\\
Auch unter Verwendung der nicht gemittelten Werte für $K$ ergeben sich Abweichung in ähnlicher Größenordnung.
\subsection{Schwebungsdauer}
Eine ca 10\%-ige Abweichung von den gemessenen Werten bei beiden Längen ist hier plausibel, da die Schwebung per Augenmaß festgestellt wurde und es weitere statistische Messfehler gab.