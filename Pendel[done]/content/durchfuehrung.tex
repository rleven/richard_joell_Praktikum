\section{Durchführung}
\label{sec:Durchführung}

Um die Messgenauigkeit der Pendel zu überprüfen, wird zunächst die Kopplungsfeder entfernt und beide Massen werden auf eine Pendellänge eingestellt.
Die Schwingungsdauer wird bei jedem Pendel einzeln gemessen.
Es werden je 9 Messungen vorgenommen, bei denen je fünf Schwingungen mit der Uhr gestoppt werden.
\\
\ 
\\
Nun wird die Feder eingebaut und es werden die Schwingungsdauern für die gleichphasige sowie gegenphasige Schwingung gemessen.
Dabei werden erneut je 9 Messungen vorgenommen, die die Dauer von fünf Schwingungen messen.
Dies wird für einmal für eine Pendellänge von $58\ cm\pm0.1\ cm$ und für $83\ cm\pm0.1\ cm$ durchgeführt.
\\
\ 
\\
Bei der gekoppelten Schwingung werden sowohl die Schwingungsdauer $T$, als auch die Schwebungsdauer $T_S$ gemessen.
Dazu werden 10 Messungen für jede Dauer aufgenommen.
Um $T$ zu erhalten werden ebenfalls die Perioden gezählt und später durch die Messung dividiert, wie in \autoref{sec:anhang} zu sehen ist.
Für $T_S$ wird je nur eine Periode für die 10 Messungen gestoppt.
\\
\ 
\\
Aus den Messungen werden nun $\omega_-,\ \omega_+\ und\ \omega_S$, sowie die Kopplungskonstante $K$ bestimmt und mit den Theoriewerten verglichen.