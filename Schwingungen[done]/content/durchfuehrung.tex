\section{Durchführung}
\label{sec:Durchführung}

Um in Erfahrung zu bringen, wie hoch der erste Widerstand \(R_1\) ist, wird mithilfe des Frequenzgenerators eine Rechteckspannung generiert.
Diese wird so gewählt, dass das Oszilloskop die abnehmende Schwingung eindeutig anzeigen kann.
Gemessen werden dann 10 einzelne Amplituden und deren Abstand zueinander.
Siehe \autoref{sec:Auswertung} für die Ermittlung von \(R_1\).
\\\\
Als nächstes ist in Erfahrung zu bringen, bei welchem Widerstand \(R_{ap}\) der aperiodische Grenzfall im Schwingkreis eintritt, wieder mit einer Rechteckspannung.
Dafür ist ein verstellbarer Widerstand am geeignetsten, da dieser fein eingestellt und das Resultat direkt am Oszillographen abgelesen werden kann.
Bei höchster Einstellung von 10.00 am Drehrad beträgt \(R_{ap\_max}\)=5k\(\Omega\), wie auf \autoref{fig:schalt} zu sehen ist. Mithilfe eines einfachen Dreisatzes kann der Widerstand demnach ermittelt werden.
Siehe \autoref{sec:Auswertung} für die Ermittlung von \(R_{ap}\).
\\\\
Um zu ermitteln in welchem Verhältnis sich Kondensatorspannung \(U_C\) und Frequenz des Eingangssignals \(\nu\) zueinander stehen, wird statt einer Recheckspannung eine Sinusspannung am Frequenzgenerator eingestellt.
Hierbei wird der größere Widerstand \(R_2\) benutzt.
Aus dem resultierenden Sinus-Signal werden 15 Werte entnommen, jeweils Amplitude mit dazugehöriger Frequenz, von 5kHz bis 75kHz in 5kHz-Schritten.
Siehe \autoref{sec:Auswertung} für die Ermittlung der Abhängigkeit. 
\\\\
Zuletzt wird der Phasenunterschied zwischen dem Sinus-Signal des Frequenzgenerators und des selben Signals nach Durchlaufen des Schwingkreises gemessen.
Beide Signale werden am Oszillographen bildlich dargestellt und deren Phasenunterschied mittels der \autoref{eq:phase} und 6 Messwerten ausgerechnet.
Es werden Frequenzwerte von 40kHz bis 140kHz in 20kHz-Schritten verwendet.