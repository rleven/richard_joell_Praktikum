\section{Diskussion}
\label{sec:Diskussion}

\subsection{Statische Methode}
Bei \autoref{fig:45grad1} und \autoref{fig:45grad2} zeigen sich logarithmische Wachstüme, wobei Aluminium am schnellsten steigt, also am besten die Wärme leitet.
Es ist ebenfalls zu erkennen, dass die Form ausschlaggebend für die Wärmeleitung ist, da sich der dünne Messingstab schneller erhitzt als der Breite.
Edelstahl steigt hingegen fast linear an, dies liegt aber nur an der kurzen Dauer der Messung.
\\
\
\\
In \autoref{fig:deltaT} sieht man, dass beide Metalle gegen eine bestimmte Temperaturdifferenz streben.
Die lokalen Maxima, die bei beiden Graphen vorliegen, sind Überschüsse der Temperatur im Metall, die durch die schlagartige Erhitzung entsteht.
Daher sind sie auch direkt zu Anfang der Messung.
Durch die hohe Kurve (im Vergleich zu Messing) ist auch zu erkennen, dass Edelstahl die Wärme nur langsam leitet, da die Temperaturdifferenz der Abtastpunkte viel größer ist, als bei Messing.
\subsection{Dynamische Methode}
Bei den drei Berechnungen für die Metalle Messing, Aluminium und Edelstahl betragen die Abweichungen zu den Literaturwerten aus \autoref{it:item}:
\begin{itemize}
    \item[] $1- \frac{\kappa_{Messing}}{\kappa_{Messing,Lit}} = 25.23\% \pm 3.6\%$
    \item[] $1- \frac{\kappa_{Aluminium}}{\kappa_{Aluminium,Lit}} = 7.17\% \pm 5.49\%$ 
    \item[] $1- \frac{\kappa_{Edelstahl}}{\kappa_{Edelstahl,Lit}} = 61.28\% \pm 1.8\%$ 
\end{itemize}
Bei Edelstahl ist die größte Abweichung, was wahrscheinlich durch systematische Fehler zu erklären ist.