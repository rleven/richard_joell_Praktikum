\section{Theorie}
\label{sec:Theorie}

Ein Temperaturungleichgewicht kann durch Wärmetransport, wie zum Beispiel Wärmestrahlung oder Konvektion, ausgeglichen werden. Dabei werden Phononen und freie Elektronen transportiert.\\
Die dabei überführte Wärmemenge in einem Stab wird wie folgt berechnet:

\begin{equation}
    dQ = -\kappa A \frac{\partial T}{\partial x} dt
    \label{eq:warmcurrent}
\end{equation}

wobei A die Querschnittsfläche des Stabes ist, L die Länge, \(\kappa\) die Wärmeleitfähigkeit, \(\frac{\partial T}{\partial x}\) die Temperaturübertragung, wobei das Vorzeichen durch die in Richtung geringerer Wärme fließender Temperatur entsteht.\\
Es gilt für die Wärmestromdichte
\begin{equation}
    j_w = -\kappa\frac{\partial T}{\partial x}
    \label{eq:jw}
\end{equation}
und die Wärmeleitungsgleichung
\begin{equation}
    \frac{\partial T}{\partial t} = \frac{\kappa}{\rho c}\frac{\partial^2 T}{\partial x^2}
    \label{eq:wavey}
\end{equation}
Dabei ist \(\rho\) die Materialdichte und c die spezifische Wärme ist.\\
Die Wärme breitet sich wellenartig aus, weshalb sie mit seiner Funktion, dessen Form die einer typischen Welle ist, beschrieben werden kann (was bereits durch die wellenförmige Differentialgleichung, \autoref{eq:wavey}, ersichtlich wird. Anhand dessen kann übrigens induziert werden, dass die Proportionalitätskonstante die Geschwindigkeit der Temperaturveränderung darstellt).:
\begin{equation}
    T(x,t) = T_{max}e^{-x\sqrt{\frac{\omega \rho c}{2\kappa}}}\cos\biggl(\omega t-x\sqrt{\frac{\omega \rho c}{2\kappa}}\biggr) 
\end{equation}
Wird diese Funktion in \autoref{eq:wavey} eingesetzt, so ergibt sich für die Phasengeschwindigkeit \(v\):
\begin{equation}
    v = \sqrt{\frac{2\kappa \omega}{\rho c}}
\end{equation}
Anhand des Amplitudenverhältnisses an zwei verschiedenen Messstellen mit Abstand \(\Delta x\) ergibt sich für \(\kappa\):
\begin{equation}
    \kappa = \frac{\rho c (\Delta x)^2}{2\ln\biggl(\frac{A_1}{A_2}\biggr)\Delta t}
    \label{eq:warleit}
\end{equation}

\(\Delta t\) ist hierbei die Phasendifferenz der Temperaturwellen im Abstand \(\Delta x\).
\\
Falls Unsicherheiten auftreten, so wird die Gauß'sche Fehlerformel benutzt, um die resultierende Unsicherheit zu berechnen:
\begin{equation}
    \Delta f = \sqrt{\sum_{i=1}^N  \Bigl(\frac{\partial f}{\partial x_i}\Bigr)^2 (\Delta x_i)^2}
    \label{eq:gauß}
\end{equation}