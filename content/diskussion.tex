\section{Diskussion}
\label{sec:Diskussion}

Der Anstieg des magnetischen Moments ist in etwa linear, was zu erwarten ist, da der magnetische Moment proportional zur magnetischen Flussdichte ist. Es wirkt eine größere Kraft, was zu Folge hat, dass sich ein magnetisches Pol schneller und stabiler richtet, als davor.
Es fällt zusätzlich auf, dass die Unsicherheit ebenfalls beträgsmäßig linear ansteigt. Das bedeutet, dass die Unsicherheit größer wird je höher die Auslenkung ist.\\
Betrachtet man nun ein Pendel, welches in einem Magnetfeld schwingt, so wird ersichtlich, dass die Periodendauer geringer wird, je stärker das Magnetfeld, in das es sich befindet, ist. Dieses Verhalten stützt das Ergebnis der Theorie für die Schwingung in einem Magnetfeld. Laut Theorie wird die Periodendauer kleiner je größer die magnetische Flussdichte ist.\\
Falls das Pendel jedoch in einer Feldfreien Zone schwingt, ist kaum eine Kraft, welche das Pendel irgendwie bremsen oder beschleunigen könnte (abgesehen von der Gravitation), vorhanden. Deshalb verändert sich die Periodendauer kaum bis gar nicht.

%Was kann man noch zum Schubmodul G, poissonsche Querkontraktionszahl µ und Kompressionsmodul Q sagen??