\section{Diskussion}
\label{sec:Diskussion}

Es kommt bei den experimentell ermittelten Werten zu riesigen Abweichungen. Folglich kann nicht verifiziert werden, ob die klassische Vorgehensweise gerechtfertigt ist. Zunächst einmal könnten die ungünstigen Bedingungen eine entscheidende Rolle spielen. Einerseits ist da der Verlust der Wärme der zu messenden Stoffe an die Umgebung zu beachten. Zusätzlich sorgt die geringe Anzahl an Messungen dafür, dass signifikante statistische Unsicherheiten nicht ausgeschlossen werden können. Durch falsche Kalibrierungen der Messgeräte oder gar technische Fehler, die teilweise im Laufe des Experiments behoben werden mussten, ist es wahrscheinlich zu systematische Fehler gekommen.\\
Ein weiterer Grund für diese großen Abweichungen könnte die Annahme sein, dass es bei der \autoref{eq:ck} und der \autoref{eq:cgmg} keine Wärmeverluste gibt, sondern jegliche Wärmemenge übertragen wird. 