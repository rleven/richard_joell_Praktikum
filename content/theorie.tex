\section{Theorie}
\label{sec:Theorie}

\subsection{Das Dulong-Petitsche Gesetz mit klassischer Physik}

Der Zusammenhang der Molwärme $C$ und der Temperatur $T$ lässt sich durch die Wärmemenge $Q$ herstellen.
Es gilt:
\begin{equation}
    C = \frac{dQ}{dT}
\end{equation}
Es bedeutet, dass sich der Stoff bei einer Temperaturerhöhung $dT$ um $dQ$ erwärmt.\\
Die Molwärme kann bei konstantem Druck, oder bei konstantem Volumen gemessen werden.
Die Differenz von $C_P$ und $C_V$ ist definiert mit:
\begin{equation}
    C_P\ -\ C_V \ =\ 9\alpha^2 \kappa V_0 T
\end{equation}
Hierbei ist $\alpha$ der lineare Ausdehnungskoeffizient, $\kappa$ ist das Kompressionsmodul, was auch durch $V(\frac{\partial\rho}{\partial V})_T$ beschrieben wird.
$V_0$ ist das Molvolumen.\\
Die Molwärme $C_P$ lässt sich ebenfalls schreiben durch:
\begin{equation}
    C_P = c_kM
    \label{eq:cp}
\end{equation}
Dabei ist $c_k$ die spezifische Wärmekapazität und $M$ die Masse des Metalls.\\
Um die spezifische Wärmekapazität zu bestimmen, werden die Temperatur des Stoffes $T_k$, die Temperatur des Wassers $T_W$, die Mischungstemperatur $T_m$, die Masse des Wassers und des Stoffes $m_k$ und $m_W$, sowie die spezifischen Wärmekapazitäten von Stoff, Wasser und Thermobehälter.
Damit berechnet man:
\begin{equation}
    c_k = \frac{(c_Wm_W\ +\ c_gm_g)(T_m-T_W)}{m_k(T_k\ -\ T_m)}
    \label{eq:warmcap}
\end{equation}
\newline
Das Dulong-Petitsche Gesetz sagt aus, dass die Molwärme eines Stoffes immer $C_V = 3R$ betragen müsste, unabhängig von den Eigenschaften des Stoffes.
Dabei ist $R$ die allgemeine Gaskonstante und beträgt ca $8.314\ \frac{kg\ m^2}{s^2\ mol\ K}$ \cite{gas}.

\subsection{Das Dulong-Petitsche Gesetz mit Quantenphysik}

Es zeigt sich, dass das Dulong-Petitsche Gesetz nicht immer korrekte Werte angibt.
Das hängt damit zusammen, dass für atomare Schwingungen die Beziehung
\begin{equation}
    \Delta E = n\cdot \hbar \cdot \omega
\end{equation}
gilt, wobei $\omega$ die Frequenz ist. Nur diese Energie kann das Atom dann abgeben oder aufnehmen.\\
Dadurch ist der quantenmechanische Zusammenhang von Energie und Temperatur nun:
\begin{equation}
    E = \frac{3N_A\hbar \omega}{e^{\frac{\hbar \omega}{kT}}\ -\ 1}
\end{equation}
Das $N_A$ ist dabei die Avogardokonstante.