\section{Theorie}
\label{sec:Theorie}

\subsection{Einzelnes Fadenpendel}
Ein Fadenpendel der Länge $l$ erfährt ein Drehmoment $M = D_p\cdot \phi$ bei seiner Auslenkung.
Dabei ist $\phi$ der Winkel der Auslenkung und $D_p$ die Winkelrichtgröße vom Pendel.\\
Ein zweites Drehmoment resultiert aus dem Trägheitsmoment $J$ der Pendelmasse $m$.
Es gilt:
\begin{equation}
    M = -J\frac{d^2\phi}{dt^2} \quad\textrm{und}\quad M = D_p\cdot \phi
\end{equation}
Daraus folgt der harmonische Oszillator für ein mathematisches Pendel unter Berücksichtigung der Kleinwinkelnäherung für $\phi$:
\begin{equation}
    J\cdot \ddot{\phi}\ +\ D_p\phi = 0
\end{equation}
Die Schwingungsfrequenz der Lösung dieser Bewegungsgleichung ist definiert als:
\begin{equation}
    \omega = \sqrt{\frac{D_p}{J}} = \sqrt{\frac{mlg}{ml^2}} = \sqrt{\frac{g}{l}}
    \label{eq:swingfreq}
\end{equation}

\subsection{Gekoppelte Fadenpendel}
Zwei Fadenpendel sind über eine Feder mit Federkonstante $D_F$ gekoppelt, sodass auf jedes Pendel je ein weiteres Drehmoment wirkt.
\begin{equation}
    M_1 = D_F(\phi_2\ -\ \phi_1) \quad\textrm{sowie}\quad M_2 = D_F(\phi_1\ -\ \phi_2)
\end{equation}
Hieraus resultiert ein Differentialgleichungssystem, welches die nun voneinander abhängigen Schwingungen beschreibt:
\begin{equation}
    J\cdot \ddot{\phi}\ +\ D_p\phi = D_F(\phi_2\ -\ \phi_1)
\end{equation}
\begin{equation}
    J\cdot \ddot{\phi}\ +\ D_p\phi = D_F(\phi_1\ -\ \phi_2)
\end{equation}
Die Auslenkwinkel $\alpha_1$ und $\alpha_2$ der beiden Pendel bestimmen die Schwingform.
Abhängig von den Anfangsbedingungen $\alpha$ und $\alpha'$ ergeben sich drei Schwingungsarten:
\subsubsection{Gleichphasige Schwingung}
%\begin{table}
    %\centering
    %\caption{Messwerte x und y sowie das Ergebnis nach Gleichung blabla.}
    %\csvreader[tabular=c|c|c,
    %head=false, 
    %table head= $x\:/\:\si{\mega\hertz}$ & $y\:/\:\si{\centi\meter}$ & $Ergebnis \:/\: \si{\milli\tesla}$ \\\midrule,
    %late after line= \\]
    %{tabelle1.csv}{1=\eins, 2=\zwei, 3=\drei}{$\num{\eins}$ & $\num{\zwei}$ & $\num{\drei}$}
    %\label{tab:messwerte}
%\end{table}
Für $\alpha_1 = \alpha_2$ resultiert eine gleichphasige Schwingung in der die Feder keine Kraft ausübt.
Da die Pendel nur durch Schwerkraft angetrieben werden, gilt für dieses System die gleiche Frequenz, wie in \autoref{eq:swingfreq}.
\begin{equation}
    \omega_+ = \sqrt{\frac{g}{l}}
\end{equation}
Dementsprechend ist die Schwingungsdauer
\begin{equation}
    T_+ = \frac{2\pi}{\omega_+} = 2\pi\ \sqrt{\frac{l}{g}}
    \label{eq:lol1}
\end{equation}

\subsubsection{Gegenphasige Schwingung}
Für $\alpha_1 = -\alpha_2$ ergibt sich eine entgegengesetzte, gleich große Schwingung der Pendel, da die Feder beide Pendel mit einer gleich großen Kraft beeinflusst.
Die Schwingungsfrequenz dafür lautet:
\begin{equation}
    \omega_- = \sqrt{\frac{g}{l}\ +\ \frac{2K}{l}}
\end{equation}
Dabei ist $K$ die Kopplungskonstante der Feder.
Die Schwingungsdauer ist somit:
\begin{equation}
    T_- = 2\pi\ \sqrt{\frac{l}{g+2K}}
    \label{eq:lol2}
\end{equation}

\subsubsection{Gekoppelte Schwingung}
Für den Fall, dass $\alpha_1 = 0,\ \alpha_2 \neq 0$, bleibt eines der Pendel in seiner Ruhelage und nur das andere wird ausgelenkt.
Daraus resultiert eine ständig wechselnde Kraftübertragung von einem Pendel auf das andere, die zu den periodischen Stillständen der Pendel führen.
Die Zeit zwischen zwei solcher Stillstände wird Schwebungsdauer genannt.\\
Sie besteht aus den Schwingungsdauern der gleichphasigen und gegenphasigen Schwingung:
\begin{equation}
    T_S = \frac{T_+\cdot T_-}{T_+\ -\ T_-}
    \label{eq:Tpose}
\end{equation}
mit der Schwebungsfrequenz
\begin{equation}
    \omega_S = \omega_-\ -\ \omega_+
    \label{eq:lol3}
\end{equation}
Die Definition für die Kopplungskonstante lautet demnach:
\begin{equation}
    K = \frac{\omega_-^2\ -\ \omega_+^2}{\omega_-^2\ +\ \omega_+^2} = \frac{T_+^2\ -\ T_-^2}{T_+^2\ +\ T_-^2}
    \label{eq:Kform}
\end{equation}

\subsection{Gauß'sche Fehlerfortpflanzung}
\label{sec:gauß}
Die bei der Auswertung verwendeten Größen wurden gemäß der Gauß'schen Fehlerfortpflanzung berechnet.
Die Formel dafür lautet:
\begin{equation}
    \overline{v_x}=\frac{1}{N} \sum_{j=1}^N v_j
\end{equation}
für den Mittelwert der Messung, und
\begin{equation}
    u_x = \sqrt{\frac{1}{N(N-1)} \sum_{j=1}^N (v_j-\overline{v_x})^2}
\end{equation}
für die Unsicherheiten der Messwerte $v_x$.