\section{Theorie}
\label{sec:Theorie}

\subsection{Das Dulong-Petitsche Gesetz mit klassischer Physik}

Der Zusammenhang der Molwärme $C$ und der Temperatur $T$ lässt sich durch die Wärmemenge $Q$ herstellen.
Es gilt:
\begin{equation}
    C = \frac{dQ}{dT}
\end{equation}
Es bedeutet, dass sich der Stoff bei einer Temperaturerhöhung $dT$ um $dQ$ erwärmt.\\
Die Molwärme kann bei konstantem Druck, oder bei konstantem Volumen gemessen werden.
Die Differenz von $C_P$ und $C_V$ ist definiert mit:
\begin{equation}
    C_P\ -\ C_V \ =\ 9\alpha^2 \kappa V_0 T
\end{equation}
Hierbei ist $\alpha$ der lineare Ausdehnungskoeffizient, $\kappa$ ist das Kompressionsmodul, was auch durch $V(\frac{\partial\rho}{\partial V})_T$ beschrieben wird.
$V_0$ ist das Molvolumen.\\
Die Molwärme $C_P$ lässt sich ebenfalls schreiben durch:
\begin{equation}
    C_P = c_kM
\end{equation}
Dabei ist $c_k$ die spezifische Wärmekapazität und $M$ die Masse des Metalls.
\\
\
\\
Das Dulong-Petitsche Gesetz sagt aus, dass die Molwärme eines Stoffes immer $3R$ betragen müsste, unabhängig von den Eigenschaften des Stoffes.
Dabei ist $R$ die allgemeine Gaskonstante.

\subsection{Das Dulong-Petitsche Gesetz mit Quantenphysik}