\section{Theorie}
\label{sec:Theorie}

In vielen Punkten ist der elektrische gekoppelte Schwingkreis seinem mechanischen Äquivalent sehr ähnlich.
So besitzt der elektrische Schwingkreis eine mitphasige und eine gegenphasige Schwingungen. 
Die "Feder" ist jedoch ein Kondensator, der die beiden Schaltkreise verbindet.
Schwingen beide Kreise mit gleicher Phase und Frequenz, wird dieser gewissermaßen übergangen, so wie auch die Feder bei gleicher Schwingung nicht ausgelenkt wird.
Sollten beide Kreise gegenphasig schwingen, so wird der Kondensator stark belastet.
\subsection{Gedämpfter Schwingkreis}
Der gemeine gedämpfte Schwingkreis besteht aus 3 Basiskomponenten: Kondensator, Spule und Widerstand.
Der Strom am geladenen Kondensator beginnt über die Spule zu fließen, sobald keine Stromquelle mehr vorliegt, bis der Kondensator seine Polarität gewechselt hat.
Der Strom erzeugt bei diesem Vorgang ein Magnetfeld um die Spule herum. 
Wenn sich die gesamte Ladung an der anderen Kondensatorplatte angelagert hat, kollabiert das Magnetfeld und sorgt dafür, dass die Ladung wieder beginnt zur anderen Platte zu fließen.
Der Widerstand wandelt dabei bei jedem Laden und Entladen des Kondensators einen Teil des Stroms in Wärme um, sodass der Schwingkreis gestört wird und nach ein paar Zyklen verebbt.
Die Differenzialgleichung für diesen Schwingkreis lautet:
\begin{equation}
    \ddot{I}(t) + \frac{R}{L}\dot{I} + \omega I(t) = 0 \quad\textrm{mit}\quad \omega = \sqrt{\frac{1}{LC}}
    \label{eq:harmossi}
\end{equation}
Die allgemeine Lösung dazu lautet:
\begin{equation}
    I(t) = A_0 \cdot e^{-\frac{R}{2L}t} \cdot \cos{(\sqrt{\frac{1}{LC}-\frac{R^2}{2L^2}}\cdot t+\eta)}
\end{equation}