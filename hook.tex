\documentclass{scrartcl}
\usepackage[aux]{rerunfilecheck}

\usepackage{fontspec}
\usepackage[main=ngerman]{babel}
\usepackage[unicode]{hyperref}
\usepackage{bookmark}
\usepackage{amsmath}
\usepackage{amssymb}
\usepackage{mathtools}
\setmainfont{Libertinus Serif}
\subject{Versuchsnummer: 803}
\title{Das Hook'sche Gesetz}
\author{Richard Leven \and Joell - D. Jones} 
\date{
    Durchführung: 05.11.2019\\
    Abgabe: 12.11.2019
}
\publishers{TUDortmund -Fachschaft Physik}
\begin{document}
\maketitle
\section{Der Versuch}
\subsection{Versuchsbeschreibung}

Es ist ein Lineal horizontal angelegt. Darauf befindet sich ein Pfeil, welcher mit einem Faden verbunden ist. Dieser Faden geht von einer Feder aus und wird von einer Rolle umgelenkt. Auf dem Bildschirm wird die ausgeübte Kraft angezeigt.

\subsection{Versuchsdurchführung}

Der Pfeil wird nun auf dem Lineal nach rechts gezogen. Der Faden zieht dementsprechend an der Feder und übt eine Gegenkraft aus, welche dann vom Laptop angezeigt wird. Dies wird dann entsprechend in einer Tabelle dokumentiert.

%.
\end{document}